% Options for packages loaded elsewhere
\PassOptionsToPackage{unicode}{hyperref}
\PassOptionsToPackage{hyphens}{url}
%
\documentclass[
]{article}
\usepackage{amsmath,amssymb}
\usepackage{lmodern}
\usepackage{iftex}
\ifPDFTeX
  \usepackage[T1]{fontenc}
  \usepackage[utf8]{inputenc}
  \usepackage{textcomp} % provide euro and other symbols
\else % if luatex or xetex
  \usepackage{unicode-math}
  \defaultfontfeatures{Scale=MatchLowercase}
  \defaultfontfeatures[\rmfamily]{Ligatures=TeX,Scale=1}
\fi
% Use upquote if available, for straight quotes in verbatim environments
\IfFileExists{upquote.sty}{\usepackage{upquote}}{}
\IfFileExists{microtype.sty}{% use microtype if available
  \usepackage[]{microtype}
  \UseMicrotypeSet[protrusion]{basicmath} % disable protrusion for tt fonts
}{}
\makeatletter
\@ifundefined{KOMAClassName}{% if non-KOMA class
  \IfFileExists{parskip.sty}{%
    \usepackage{parskip}
  }{% else
    \setlength{\parindent}{0pt}
    \setlength{\parskip}{6pt plus 2pt minus 1pt}}
}{% if KOMA class
  \KOMAoptions{parskip=half}}
\makeatother
\usepackage{xcolor}
\usepackage[margin=1in]{geometry}
\usepackage{color}
\usepackage{fancyvrb}
\newcommand{\VerbBar}{|}
\newcommand{\VERB}{\Verb[commandchars=\\\{\}]}
\DefineVerbatimEnvironment{Highlighting}{Verbatim}{commandchars=\\\{\}}
% Add ',fontsize=\small' for more characters per line
\usepackage{framed}
\definecolor{shadecolor}{RGB}{248,248,248}
\newenvironment{Shaded}{\begin{snugshade}}{\end{snugshade}}
\newcommand{\AlertTok}[1]{\textcolor[rgb]{0.94,0.16,0.16}{#1}}
\newcommand{\AnnotationTok}[1]{\textcolor[rgb]{0.56,0.35,0.01}{\textbf{\textit{#1}}}}
\newcommand{\AttributeTok}[1]{\textcolor[rgb]{0.77,0.63,0.00}{#1}}
\newcommand{\BaseNTok}[1]{\textcolor[rgb]{0.00,0.00,0.81}{#1}}
\newcommand{\BuiltInTok}[1]{#1}
\newcommand{\CharTok}[1]{\textcolor[rgb]{0.31,0.60,0.02}{#1}}
\newcommand{\CommentTok}[1]{\textcolor[rgb]{0.56,0.35,0.01}{\textit{#1}}}
\newcommand{\CommentVarTok}[1]{\textcolor[rgb]{0.56,0.35,0.01}{\textbf{\textit{#1}}}}
\newcommand{\ConstantTok}[1]{\textcolor[rgb]{0.00,0.00,0.00}{#1}}
\newcommand{\ControlFlowTok}[1]{\textcolor[rgb]{0.13,0.29,0.53}{\textbf{#1}}}
\newcommand{\DataTypeTok}[1]{\textcolor[rgb]{0.13,0.29,0.53}{#1}}
\newcommand{\DecValTok}[1]{\textcolor[rgb]{0.00,0.00,0.81}{#1}}
\newcommand{\DocumentationTok}[1]{\textcolor[rgb]{0.56,0.35,0.01}{\textbf{\textit{#1}}}}
\newcommand{\ErrorTok}[1]{\textcolor[rgb]{0.64,0.00,0.00}{\textbf{#1}}}
\newcommand{\ExtensionTok}[1]{#1}
\newcommand{\FloatTok}[1]{\textcolor[rgb]{0.00,0.00,0.81}{#1}}
\newcommand{\FunctionTok}[1]{\textcolor[rgb]{0.00,0.00,0.00}{#1}}
\newcommand{\ImportTok}[1]{#1}
\newcommand{\InformationTok}[1]{\textcolor[rgb]{0.56,0.35,0.01}{\textbf{\textit{#1}}}}
\newcommand{\KeywordTok}[1]{\textcolor[rgb]{0.13,0.29,0.53}{\textbf{#1}}}
\newcommand{\NormalTok}[1]{#1}
\newcommand{\OperatorTok}[1]{\textcolor[rgb]{0.81,0.36,0.00}{\textbf{#1}}}
\newcommand{\OtherTok}[1]{\textcolor[rgb]{0.56,0.35,0.01}{#1}}
\newcommand{\PreprocessorTok}[1]{\textcolor[rgb]{0.56,0.35,0.01}{\textit{#1}}}
\newcommand{\RegionMarkerTok}[1]{#1}
\newcommand{\SpecialCharTok}[1]{\textcolor[rgb]{0.00,0.00,0.00}{#1}}
\newcommand{\SpecialStringTok}[1]{\textcolor[rgb]{0.31,0.60,0.02}{#1}}
\newcommand{\StringTok}[1]{\textcolor[rgb]{0.31,0.60,0.02}{#1}}
\newcommand{\VariableTok}[1]{\textcolor[rgb]{0.00,0.00,0.00}{#1}}
\newcommand{\VerbatimStringTok}[1]{\textcolor[rgb]{0.31,0.60,0.02}{#1}}
\newcommand{\WarningTok}[1]{\textcolor[rgb]{0.56,0.35,0.01}{\textbf{\textit{#1}}}}
\usepackage{graphicx}
\makeatletter
\def\maxwidth{\ifdim\Gin@nat@width>\linewidth\linewidth\else\Gin@nat@width\fi}
\def\maxheight{\ifdim\Gin@nat@height>\textheight\textheight\else\Gin@nat@height\fi}
\makeatother
% Scale images if necessary, so that they will not overflow the page
% margins by default, and it is still possible to overwrite the defaults
% using explicit options in \includegraphics[width, height, ...]{}
\setkeys{Gin}{width=\maxwidth,height=\maxheight,keepaspectratio}
% Set default figure placement to htbp
\makeatletter
\def\fps@figure{htbp}
\makeatother
\setlength{\emergencystretch}{3em} % prevent overfull lines
\providecommand{\tightlist}{%
  \setlength{\itemsep}{0pt}\setlength{\parskip}{0pt}}
\setcounter{secnumdepth}{-\maxdimen} % remove section numbering
\ifLuaTeX
  \usepackage{selnolig}  % disable illegal ligatures
\fi
\IfFileExists{bookmark.sty}{\usepackage{bookmark}}{\usepackage{hyperref}}
\IfFileExists{xurl.sty}{\usepackage{xurl}}{} % add URL line breaks if available
\urlstyle{same} % disable monospaced font for URLs
\hypersetup{
  pdftitle={Forecasting: Principles and Practice -- 2nd Edition},
  pdfauthor={Dibz},
  hidelinks,
  pdfcreator={LaTeX via pandoc}}

\title{Forecasting: Principles and Practice -- 2nd Edition}
\author{Dibz}
\date{2023-04-01}

\begin{document}
\maketitle

\hypertarget{chapter-2---time-series-graphics}{%
\subsection{Chapter 2 - Time series
graphics}\label{chapter-2---time-series-graphics}}

\hypertarget{ts-objects}{%
\section{\texorpdfstring{2.1 \(ts\)
objects}{2.1 ts objects}}\label{ts-objects}}

\(ts\) function turn data to time series

\begin{Shaded}
\begin{Highlighting}[]
\NormalTok{y }\OtherTok{\textless{}{-}} \FunctionTok{ts}\NormalTok{(}\FunctionTok{c}\NormalTok{(}\DecValTok{123}\NormalTok{,}\DecValTok{39}\NormalTok{,}\DecValTok{78}\NormalTok{,}\DecValTok{52}\NormalTok{,}\DecValTok{110}\NormalTok{),}\AttributeTok{start =} \DecValTok{2012}\NormalTok{)}
\end{Highlighting}
\end{Shaded}

Suppose we have vector z, we can write

\begin{Shaded}
\begin{Highlighting}[]
\NormalTok{z }\OtherTok{\textless{}{-}} \FunctionTok{runif}\NormalTok{(}\DecValTok{24}\NormalTok{,}\DecValTok{1}\NormalTok{,}\DecValTok{100}\NormalTok{)}
\NormalTok{y }\OtherTok{\textless{}{-}} \FunctionTok{ts}\NormalTok{(z, }\AttributeTok{start =} \DecValTok{2003}\NormalTok{, }\AttributeTok{frequency =} \DecValTok{12}\NormalTok{)}
\end{Highlighting}
\end{Shaded}

\hypertarget{time-plots}{%
\section{2.2 Time plots}\label{time-plots}}

Examples

\begin{Shaded}
\begin{Highlighting}[]
\FunctionTok{library}\NormalTok{(fpp2)}
\end{Highlighting}
\end{Shaded}

\begin{verbatim}
## Registered S3 method overwritten by 'quantmod':
##   method            from
##   as.zoo.data.frame zoo
\end{verbatim}

\begin{verbatim}
## -- Attaching packages ---------------------------------------------- fpp2 2.5 --
\end{verbatim}

\begin{verbatim}
## v ggplot2   3.4.0     v fma       2.5  
## v forecast  8.21      v expsmooth 2.3
\end{verbatim}

\begin{verbatim}
## 
\end{verbatim}

\begin{Shaded}
\begin{Highlighting}[]
\FunctionTok{autoplot}\NormalTok{(melsyd[,}\StringTok{"Economy.Class"}\NormalTok{]) }\SpecialCharTok{+}
  \FunctionTok{ggtitle}\NormalTok{(}\StringTok{"Economy class passengers: Melbourne{-}Sydney"}\NormalTok{) }\SpecialCharTok{+}
  \FunctionTok{xlab}\NormalTok{(}\StringTok{"Year"}\NormalTok{) }\SpecialCharTok{+}
  \FunctionTok{ylab}\NormalTok{(}\StringTok{"Thousands"}\NormalTok{)}
\end{Highlighting}
\end{Shaded}

\includegraphics{Notes-on-Forecasting_230401_files/figure-latex/unnamed-chunk-3-1.pdf}

\begin{Shaded}
\begin{Highlighting}[]
\FunctionTok{autoplot}\NormalTok{(a10) }\SpecialCharTok{+}
  \FunctionTok{ggtitle}\NormalTok{(}\StringTok{"Antidiabetic drug sales"}\NormalTok{) }\SpecialCharTok{+}
  \FunctionTok{ylab}\NormalTok{(}\StringTok{"$ million"}\NormalTok{) }\SpecialCharTok{+}
  \FunctionTok{xlab}\NormalTok{(}\StringTok{"Year"}\NormalTok{)}
\end{Highlighting}
\end{Shaded}

\includegraphics{Notes-on-Forecasting_230401_files/figure-latex/unnamed-chunk-4-1.pdf}

\hypertarget{times-series-patterns}{%
\section{2.3 Times series patterns}\label{times-series-patterns}}

\textbf{Trend} - long term increase or decrease \textbf{Seasonal} -
pattern occurs when data is affected by seasonal factors. Usually, fix
and known frequency. \textbf{Cyclic} - pattern occurs not in fix
frequency and usually at least 2 years.

\hypertarget{seasonal-plots}{%
\section{2.4 Seasonal plots}\label{seasonal-plots}}

To identify seasonality

Examples

\textbf{Seasonal plot}

\begin{Shaded}
\begin{Highlighting}[]
\FunctionTok{ggseasonplot}\NormalTok{(a10, }\AttributeTok{year.labels=}\ConstantTok{TRUE}\NormalTok{, }\AttributeTok{year.labels.left=}\ConstantTok{TRUE}\NormalTok{) }\SpecialCharTok{+}
  \FunctionTok{ylab}\NormalTok{(}\StringTok{"$ million"}\NormalTok{) }\SpecialCharTok{+}
  \FunctionTok{ggtitle}\NormalTok{(}\StringTok{"Seasonal plot: antidiabetic drug sales"}\NormalTok{)}
\end{Highlighting}
\end{Shaded}

\includegraphics{Notes-on-Forecasting_230401_files/figure-latex/unnamed-chunk-5-1.pdf}

\textbf{Polar plot}

\begin{Shaded}
\begin{Highlighting}[]
\FunctionTok{ggseasonplot}\NormalTok{(a10, }\AttributeTok{polar=}\ConstantTok{TRUE}\NormalTok{) }\SpecialCharTok{+}
  \FunctionTok{ylab}\NormalTok{(}\StringTok{"$ million"}\NormalTok{) }\SpecialCharTok{+}
  \FunctionTok{ggtitle}\NormalTok{(}\StringTok{"Polar seasonal plot: antidiabetic drug sales"}\NormalTok{)}
\end{Highlighting}
\end{Shaded}

\includegraphics{Notes-on-Forecasting_230401_files/figure-latex/unnamed-chunk-6-1.pdf}

\textbf{Seasonal subseries plot}

\begin{Shaded}
\begin{Highlighting}[]
\FunctionTok{ggsubseriesplot}\NormalTok{(a10) }\SpecialCharTok{+}
  \FunctionTok{ylab}\NormalTok{(}\StringTok{"$ million"}\NormalTok{) }\SpecialCharTok{+}
  \FunctionTok{ggtitle}\NormalTok{(}\StringTok{"Seasonal subseries plot: antidiabetic drug sales"}\NormalTok{)}
\end{Highlighting}
\end{Shaded}

\includegraphics{Notes-on-Forecasting_230401_files/figure-latex/unnamed-chunk-7-1.pdf}

\hypertarget{scatter-plot}{%
\section{2.6 Scatter plot}\label{scatter-plot}}

Use to explore relationship between time series.

Examples

Data to time

\begin{Shaded}
\begin{Highlighting}[]
\FunctionTok{autoplot}\NormalTok{(elecdemand[,}\FunctionTok{c}\NormalTok{(}\StringTok{"Demand"}\NormalTok{,}\StringTok{"Temperature"}\NormalTok{)], }\AttributeTok{facets=}\ConstantTok{TRUE}\NormalTok{) }\SpecialCharTok{+}
  \FunctionTok{xlab}\NormalTok{(}\StringTok{"Year: 2014"}\NormalTok{) }\SpecialCharTok{+} \FunctionTok{ylab}\NormalTok{(}\StringTok{""}\NormalTok{) }\SpecialCharTok{+}
  \FunctionTok{ggtitle}\NormalTok{(}\StringTok{"Half{-}hourly electricity demand: Victoria, Australia"}\NormalTok{)}
\end{Highlighting}
\end{Shaded}

\includegraphics{Notes-on-Forecasting_230401_files/figure-latex/unnamed-chunk-8-1.pdf}

Variable to variable

\begin{Shaded}
\begin{Highlighting}[]
\FunctionTok{qplot}\NormalTok{(Temperature, Demand, }\AttributeTok{data=}\FunctionTok{as.data.frame}\NormalTok{(elecdemand)) }\SpecialCharTok{+}
  \FunctionTok{ylab}\NormalTok{(}\StringTok{"Demand (GW)"}\NormalTok{) }\SpecialCharTok{+} \FunctionTok{xlab}\NormalTok{(}\StringTok{"Temperature (Celsius)"}\NormalTok{)}
\end{Highlighting}
\end{Shaded}

\begin{verbatim}
## Warning: `qplot()` was deprecated in ggplot2 3.4.0.
## This warning is displayed once every 8 hours.
## Call `lifecycle::last_lifecycle_warnings()` to see where this warning was
## generated.
\end{verbatim}

\includegraphics{Notes-on-Forecasting_230401_files/figure-latex/unnamed-chunk-9-1.pdf}

\begin{Shaded}
\begin{Highlighting}[]
\CommentTok{\#\textgreater{} Warning: \textasciigrave{}qplot()\textasciigrave{} was deprecated in ggplot2 3.4.0.}
\CommentTok{\#\textgreater{} This warning is displayed once every 8 hours.}
\CommentTok{\#\textgreater{} Call \textasciigrave{}lifecycle::last\_lifecycle\_warnings()\textasciigrave{} to see where this warning was generated.}
\end{Highlighting}
\end{Shaded}

Multivariable to time

\begin{Shaded}
\begin{Highlighting}[]
\FunctionTok{autoplot}\NormalTok{(visnights[,}\DecValTok{1}\SpecialCharTok{:}\DecValTok{5}\NormalTok{], }\AttributeTok{facets=}\ConstantTok{TRUE}\NormalTok{) }\SpecialCharTok{+}
  \FunctionTok{ylab}\NormalTok{(}\StringTok{"Number of visitor nights each quarter (millions)"}\NormalTok{)}
\end{Highlighting}
\end{Shaded}

\includegraphics{Notes-on-Forecasting_230401_files/figure-latex/unnamed-chunk-10-1.pdf}

\begin{Shaded}
\begin{Highlighting}[]
\NormalTok{GGally}\SpecialCharTok{::}\FunctionTok{ggpairs}\NormalTok{(}\FunctionTok{as.data.frame}\NormalTok{(visnights[,}\DecValTok{1}\SpecialCharTok{:}\DecValTok{5}\NormalTok{]))}
\end{Highlighting}
\end{Shaded}

\begin{verbatim}
## Registered S3 method overwritten by 'GGally':
##   method from   
##   +.gg   ggplot2
\end{verbatim}

\includegraphics{Notes-on-Forecasting_230401_files/figure-latex/unnamed-chunk-11-1.pdf}

\textbf{Correlation} Linear relationship between variables.

\(r = \frac{\sum(x_t-\bar{x})(y_t-\bar{y})}{\sqrt{\sum(x_t-\bar{x})^2}{\sqrt{\sum(y_t-\bar{y})^2}}}\)

\hypertarget{lag-plots}{%
\section{2.7 Lag plots}\label{lag-plots}}

\begin{Shaded}
\begin{Highlighting}[]
\NormalTok{beer2 }\OtherTok{\textless{}{-}} \FunctionTok{window}\NormalTok{(ausbeer, }\AttributeTok{start=}\DecValTok{1992}\NormalTok{)}
\FunctionTok{gglagplot}\NormalTok{(beer2)}
\end{Highlighting}
\end{Shaded}

\includegraphics{Notes-on-Forecasting_230401_files/figure-latex/unnamed-chunk-12-1.pdf}

\hypertarget{autocorrelation}{%
\section{2.8 Autocorrelation}\label{autocorrelation}}

Correlation of the time series data with its \emph{lagged values}. \(T\)
is the lenght of the time series.

\(r_k = \frac{\sum_{t=k+1}^T(y_t-\bar{y})(y_{t-k}-\bar{y})}{\sum_{t=1}^T(y_t-\bar{y})^2}\)

Example

\begin{Shaded}
\begin{Highlighting}[]
\FunctionTok{ggAcf}\NormalTok{(beer2)}
\end{Highlighting}
\end{Shaded}

\includegraphics{Notes-on-Forecasting_230401_files/figure-latex/unnamed-chunk-13-1.pdf}

In an ACF, trended time series tend to have positive values that slowly
decrease as the lag increase. When it's seasonal, ACF is larger on the
seaonal lags. If both, there is a combination of pattern from trend and
season.

\begin{Shaded}
\begin{Highlighting}[]
\NormalTok{aelec }\OtherTok{\textless{}{-}} \FunctionTok{window}\NormalTok{(elec, }\AttributeTok{start=}\DecValTok{1980}\NormalTok{)}
\FunctionTok{autoplot}\NormalTok{(aelec) }\SpecialCharTok{+} \FunctionTok{xlab}\NormalTok{(}\StringTok{"Year"}\NormalTok{) }\SpecialCharTok{+} \FunctionTok{ylab}\NormalTok{(}\StringTok{"GWh"}\NormalTok{)}
\end{Highlighting}
\end{Shaded}

\includegraphics{Notes-on-Forecasting_230401_files/figure-latex/unnamed-chunk-14-1.pdf}

\begin{Shaded}
\begin{Highlighting}[]
\FunctionTok{ggAcf}\NormalTok{(aelec, }\AttributeTok{lag=}\DecValTok{48}\NormalTok{)}
\end{Highlighting}
\end{Shaded}

\includegraphics{Notes-on-Forecasting_230401_files/figure-latex/unnamed-chunk-15-1.pdf}

\hypertarget{white-noise}{%
\section{2.9 White noise}\label{white-noise}}

If the time series show no autocorrelation. Not necessarily equal to
zero, we expect at least 95\% of the spikes in ACF plot to lie within
\(\pm2/\sqrt{T}\).

Example

\begin{Shaded}
\begin{Highlighting}[]
\FunctionTok{set.seed}\NormalTok{(}\DecValTok{30}\NormalTok{)}
\NormalTok{y }\OtherTok{\textless{}{-}} \FunctionTok{ts}\NormalTok{(}\FunctionTok{rnorm}\NormalTok{(}\DecValTok{50}\NormalTok{))}
\FunctionTok{autoplot}\NormalTok{(y) }\SpecialCharTok{+} \FunctionTok{ggtitle}\NormalTok{(}\StringTok{"White noise"}\NormalTok{)}
\end{Highlighting}
\end{Shaded}

\includegraphics{Notes-on-Forecasting_230401_files/figure-latex/unnamed-chunk-16-1.pdf}

\begin{Shaded}
\begin{Highlighting}[]
\FunctionTok{ggAcf}\NormalTok{(y)}
\end{Highlighting}
\end{Shaded}

\includegraphics{Notes-on-Forecasting_230401_files/figure-latex/unnamed-chunk-17-1.pdf}

\end{document}
